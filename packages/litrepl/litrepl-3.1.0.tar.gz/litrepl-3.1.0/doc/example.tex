\documentclass{article}
\usepackage{verbatim}
\usepackage[utf8]{inputenc}
\begin{document}

\section{Basics}

LitREPL treats \texttt{lcode} environment as code sections and \texttt{lresult}
environment as result sections. The names are currently hardcoded into the
simplified LitREPL parser. Wrapping it in other tags is not allowed.

LaTeX does not know anything about these environments by default, we need to
introduce them to be able to compile the document using e.g. \texttt{pdflatex}.

\newenvironment{lcode}{\verbatim}{\endverbatim}
\newenvironment{lresult}{\verbatim}{\endverbatim}
\newcommand{\linline}[2]{#2}

Executable section is the text between the \texttt{lcode} begin/end tags.

\begin{lcode}
W='Hello, World!'
print(W)
\end{lcode}

Putting the cursor on it and typing the \texttt{:LEval1} runs the code in the
background Python interpreter.

\texttt{lresult} begin/end tags mark the result section.  LitREPL replaces its
content with the above code section's execution result.

\begin{lresult}
Hello, World!
\end{lresult}

\section{Producing LaTeX}

LitREPL recognizes \texttt{l[no]code}/\texttt{l[no]result} comments as
code/result section markers. This way we can use Python to produce LaTeX markup
as output.

\begin{lcode}
print("\\textbf{Hi}")
\end{lcode}

%lresult
\textbf{Hi}
%lnoresult

\section{Inline output}

Additionally, VimREPL recognises \texttt{linline} 2-argument tags. The first
arguement is treaten as a Python printable expression. The second arguemnt is to
be replaced with the printing output. In our simplified definition, we simply
ignore the first argument and paste the second to the LaTeX processor as-is.

\linline{W}{Hello, World!}

\end{document}
