\documentclass{article}
\usepackage{verbatim}
\usepackage{mdframed}
\usepackage{minted}
\renewcommand{\MintedPygmentize}{pygmentize}
\usepackage[utf8]{inputenc}

% LitREPL-compatible environment for code snippets
\newenvironment{python}
  {\VerbatimEnvironment
   \begin{minted}[breaklines,fontsize=\footnotesize]{python}}
  {\end{minted}}
\BeforeBeginEnvironment{python}{
  \begin{mdframed}[nobreak=true,frametitle=\tiny{Python}]}
\AfterEndEnvironment{python}{\end{mdframed}}

% LitREPL-compatible environment for code results
\newenvironment{result}{\verbatim}{\endverbatim}
\BeforeBeginEnvironment{result}{
  \begin{mdframed}[frametitle=\tiny{Result}]\footnotesize}
\AfterEndEnvironment{result}{\end{mdframed}}

% LitREPL-compatible command for inline code results
\newcommand{\linline}[2]{#2}

\begin{document}

\section{Preamble}

Literpl employs its own internal parser to identify LaTeX-like tags that define
code and result sections. One option is to use the \texttt{python} environment
to mark code and wrap results with \texttt{result}. Both environments need to be
defined in LaTeX so it doesn't get confused. The following preamble sets up both
environments to render as framed boxes of fixed-width text with proper
highlighting:

\begin{verbatim}
\newenvironment{python}
  {\VerbatimEnvironment
   \begin{minted}[breaklines,fontsize=\footnotesize]{python}}
  {\end{minted}}
\BeforeBeginEnvironment{python}{
  \begin{mdframed}[nobreak=true,frametitle=\tiny{Python}]}
\AfterEndEnvironment{python}{\end{mdframed}}

\newenvironment{result}{\verbatim}{\endverbatim}
\BeforeBeginEnvironment{result}{
  \begin{mdframed}[frametitle=\tiny{Result}]\footnotesize}
\AfterEndEnvironment{result}{\end{mdframed}}
\end{verbatim}


\section{Basic evaluation}

Here is how we can use the environments we just introduced.

\begin{verbatim}
\begin{python}
W='Hello, World!'
print(W)
\end{python}
\end{verbatim}

\begin{python}
W='Hello, World!'
print(W)
\end{python}

\begin{verbatim}
\begin{result}
Hello, World!
\end{result}
\end{verbatim}

\begin{result}
Hello, World!
\end{result}

\section{Producing LaTeX}

Literpl also recognizes \texttt{result} and \texttt{noresult} comments, which
serve as markers for result sections. By using these comments, we can generate
LaTeX markup directly as output.

\begin{verbatim}
\begin{python}
print("\\textbf{Hi!}")
\end{python}

%result
\textbf{Hi!}
%noresult
\end{verbatim}


\begin{python}
print("\\textbf{Hi!}")
\end{python}

%result
\textbf{Hi!}
%noresult

\section{Inline output}

Furthermore, Literpl recognizes \texttt{linline} tags with two arguments. The
first argument is treated as a Python printable expression, while the second
argument will be replaced with its evaluated value.

The value of \texttt{W} happens to be: \linline{W}{Hello, World!}

\end{document}
